
%% bare_conf.tex
%% V1.3
%% 2007/01/11
%% by Michael Shell
%% See:
%% http://www.michaelshell.org/
%% for current contact information.
%%
%% This is a skeleton file demonstrating the use of IEEEtran.cls
%% (requires IEEEtran.cls version 1.7 or later) with an IEEE conference paper.
%%
%% Support sites:
%% http://www.michaelshell.org/tex/ieeetran/
%% http://www.ctan.org/tex-archive/macros/latex/contrib/IEEEtran/
%% and
%% http://www.ieee.org/

%%*************************************************************************
%% Legal Notice:
%% This code is offered as-is without any warranty either expressed or
%% implied; without even the implied warranty of MERCHANTABILITY or
%% FITNESS FOR A PARTICULAR PURPOSE! 
%% User assumes all risk.
%% In no event shall IEEE or any contributor to this code be liable for
%% any damages or losses, including, but not limited to, incidental,
%% consequential, or any other damages, resulting from the use or misuse
%% of any information contained here.
%%
%% All comments are the opinions of their respective authors and are not
%% necessarily endorsed by the IEEE.
%%
%% This work is distributed under the LaTeX Project Public License (LPPL)
%% ( http://www.latex-project.org/ ) version 1.3, and may be freely used,
%% distributed and modified. A copy of the LPPL, version 1.3, is included
%% in the base LaTeX documentation of all distributions of LaTeX released
%% 2003/12/01 or later.
%% Retain all contribution notices and credits.
%% ** Modified files should be clearly indicated as such, including  **
%% ** renaming them and changing author support contact information. **
%%
%% File list of work: IEEEtran.cls, IEEEtran_HOWTO.pdf, bare_adv.tex,
%%                    bare_conf.tex, bare_jrnl.tex, bare_jrnl_compsoc.tex
%%*************************************************************************

% *** Authors should verify (and, if needed, correct) their LaTeX system  ***
% *** with the testflow diagnostic prior to trusting their LaTeX platform ***
% *** with production work. IEEE's font choices can trigger bugs that do  ***
% *** not appear when using other class files.                            ***
% The testflow support page is at:
% http://www.michaelshell.org/tex/testflow/



% Note that the a4paper option is mainly intended so that authors in
% countries using A4 can easily print to A4 and see how their papers will
% look in print - the typesetting of the document will not typically be
% affected with changes in paper size (but the bottom and side margins will).
% Use the testflow package mentioned above to verify correct handling of
% both paper sizes by the user's LaTeX system.
%
% Also note that the "draftcls" or "draftclsnofoot", not "draft", option
% should be used if it is desired that the figures are to be displayed in
% draft mode.
%
\documentclass[conference]{IEEEtran}
% Add the compsoc option for Computer Society conferences.
%
% If IEEEtran.cls has not been installed into the LaTeX system files,
% manually specify the path to it like:
% \documentclass[conference]{../sty/IEEEtran}

% Some very useful LaTeX packages include:
% (uncomment the ones you want to load)

% *** MISC UTILITY PACKAGES ***
%
%\usepackage{ifpdf}
% Heiko Oberdiek's ifpdf.sty is very useful if you need conditional
% compilation based on whether the output is pdf or dvi.
% usage:
% \ifpdf
%   % pdf code
% \else
%   % dvi code
% \fi
% The latest version of ifpdf.sty can be obtained from:
% http://www.ctan.org/tex-archive/macros/latex/contrib/oberdiek/
% Also, note that IEEEtran.cls V1.7 and later provides a builtin
% \ifCLASSINFOpdf conditional that works the same way.
% When switching from latex to pdflatex and vice-versa, the compiler may
% have to be run twice to clear warning/error messages.






% *** CITATION PACKAGES ***
%
\usepackage{cite}
% cite.sty was written by Donald Arseneau
% V1.6 and later of IEEEtran pre-defines the format of the cite.sty package
% \cite{} output to follow that of IEEE. Loading the cite package will
% result in citation numbers being automatically sorted and properly
% "compressed/ranged". e.g., [1], [9], [2], [7], [5], [6] without using
% cite.sty will become [1], [2], [5]--[7], [9] using cite.sty. cite.sty's
% \cite will automatically add leading space, if needed. Use cite.sty's
% noadjust option (cite.sty V3.8 and later) if you want to turn this off.
% cite.sty is already installed on most LaTeX systems. Be sure and use
% version 4.0 (2003-05-27) and later if using hyperref.sty. cite.sty does
% not currently provide for hyperlinked citations.
% The latest version can be obtained at:
% http://www.ctan.org/tex-archive/macros/latex/contrib/cite/
% The documentation is contained in the cite.sty file itself.






% *** GRAPHICS RELATED PACKAGES ***
%
\ifCLASSINFOpdf
  % \usepackage[pdftex]{graphicx}
  % declare the path(s) where your graphic files are
  % \graphicspath{{../pdf/}{../jpeg/}}
  % and their extensions so you won't have to specify these with
  % every instance of \includegraphics
  % \DeclareGraphicsExtensions{.pdf,.jpeg,.png}
\else
  % or other class option (dvipsone, dvipdf, if not using dvips). graphicx
  % will default to the driver specified in the system graphics.cfg if no
  % driver is specified.
  % \usepackage[dvips]{graphicx}
  % declare the path(s) where your graphic files are
  % \graphicspath{{../eps/}}
  % and their extensions so you won't have to specify these with
  % every instance of \includegraphics
  % \DeclareGraphicsExtensions{.eps}
\fi
% graphicx was written by David Carlisle and Sebastian Rahtz. It is
% required if you want graphics, photos, etc. graphicx.sty is already
% installed on most LaTeX systems. The latest version and documentation can
% be obtained at: 
% http://www.ctan.org/tex-archive/macros/latex/required/graphics/
% Another good source of documentation is "Using Imported Graphics in
% LaTeX2e" by Keith Reckdahl which can be found as epslatex.ps or
% epslatex.pdf at: http://www.ctan.org/tex-archive/info/
%
% latex, and pdflatex in dvi mode, support graphics in encapsulated
% postscript (.eps) format. pdflatex in pdf mode supports graphics
% in .pdf, .jpeg, .png and .mps (metapost) formats. Users should ensure
% that all non-photo figures use a vector format (.eps, .pdf, .mps) and
% not a bitmapped formats (.jpeg, .png). IEEE frowns on bitmapped formats
% which can result in "jaggedy"/blurry rendering of lines and letters as
% well as large increases in file sizes.
%
% You can find documentation about the pdfTeX application at:
% http://www.tug.org/applications/pdftex





% *** MATH PACKAGES ***
%
%\usepackage[cmex10]{amsmath}
% A popular package from the American Mathematical Society that provides
% many useful and powerful commands for dealing with mathematics. If using
% it, be sure to load this package with the cmex10 option to ensure that
% only type 1 fonts will utilized at all point sizes. Without this option,
% it is possible that some math symbols, particularly those within
% footnotes, will be rendered in bitmap form which will result in a
% document that can not be IEEE Xplore compliant!
%
% Also, note that the amsmath package sets \interdisplaylinepenalty to 10000
% thus preventing page breaks from occurring within multiline equations. Use:
%\interdisplaylinepenalty=2500
% after loading amsmath to restore such page breaks as IEEEtran.cls normally
% does. amsmath.sty is already installed on most LaTeX systems. The latest
% version and documentation can be obtained at:
% http://www.ctan.org/tex-archive/macros/latex/required/amslatex/math/





% *** SPECIALIZED LIST PACKAGES ***
%
%\usepackage{algorithmic}
% algorithmic.sty was written by Peter Williams and Rogerio Brito.
% This package provides an algorithmic environment fo describing algorithms.
% You can use the algorithmic environment in-text or within a figure
% environment to provide for a floating algorithm. Do NOT use the algorithm
% floating environment provided by algorithm.sty (by the same authors) or
% algorithm2e.sty (by Christophe Fiorio) as IEEE does not use dedicated
% algorithm float types and packages that provide these will not provide
% correct IEEE style captions. The latest version and documentation of
% algorithmic.sty can be obtained at:
% http://www.ctan.org/tex-archive/macros/latex/contrib/algorithms/
% There is also a support site at:
% http://algorithms.berlios.de/index.html
% Also of interest may be the (relatively newer and more customizable)
% algorithmicx.sty package by Szasz Janos:
% http://www.ctan.org/tex-archive/macros/latex/contrib/algorithmicx/




% *** ALIGNMENT PACKAGES ***
%
%\usepackage{array}
% Frank Mittelbach's and David Carlisle's array.sty patches and improves
% the standard LaTeX2e array and tabular environments to provide better
% appearance and additional user controls. As the default LaTeX2e table
% generation code is lacking to the point of almost being broken with
% respect to the quality of the end results, all users are strongly
% advised to use an enhanced (at the very least that provided by array.sty)
% set of table tools. array.sty is already installed on most systems. The
% latest version and documentation can be obtained at:
% http://www.ctan.org/tex-archive/macros/latex/required/tools/


%\usepackage{mdwmath}
%\usepackage{mdwtab}
% Also highly recommended is Mark Wooding's extremely powerful MDW tools,
% especially mdwmath.sty and mdwtab.sty which are used to format equations
% and tables, respectively. The MDWtools set is already installed on most
% LaTeX systems. The lastest version and documentation is available at:
% http://www.ctan.org/tex-archive/macros/latex/contrib/mdwtools/


% IEEEtran contains the IEEEeqnarray family of commands that can be used to
% generate multiline equations as well as matrices, tables, etc., of high
% quality.


%\usepackage{eqparbox}
% Also of notable interest is Scott Pakin's eqparbox package for creating
% (automatically sized) equal width boxes - aka "natural width parboxes".
% Available at:
% http://www.ctan.org/tex-archive/macros/latex/contrib/eqparbox/





% *** SUBFIGURE PACKAGES ***
%\usepackage[tight,footnotesize]{subfigure}
% subfigure.sty was written by Steven Douglas Cochran. This package makes it
% easy to put subfigures in your figures. e.g., "Figure 1a and 1b". For IEEE
% work, it is a good idea to load it with the tight package option to reduce
% the amount of white space around the subfigures. subfigure.sty is already
% installed on most LaTeX systems. The latest version and documentation can
% be obtained at:
% http://www.ctan.org/tex-archive/obsolete/macros/latex/contrib/subfigure/
% subfigure.sty has been superceeded by subfig.sty.



%\usepackage[caption=false]{caption}
%\usepackage[font=footnotesize]{subfig}
% subfig.sty, also written by Steven Douglas Cochran, is the modern
% replacement for subfigure.sty. However, subfig.sty requires and
% automatically loads Axel Sommerfeldt's caption.sty which will override
% IEEEtran.cls handling of captions and this will result in nonIEEE style
% figure/table captions. To prevent this problem, be sure and preload
% caption.sty with its "caption=false" package option. This is will preserve
% IEEEtran.cls handing of captions. Version 1.3 (2005/06/28) and later 
% (recommended due to many improvements over 1.2) of subfig.sty supports
% the caption=false option directly:
%\usepackage[caption=false,font=footnotesize]{subfig}
%
% The latest version and documentation can be obtained at:
% http://www.ctan.org/tex-archive/macros/latex/contrib/subfig/
% The latest version and documentation of caption.sty can be obtained at:
% http://www.ctan.org/tex-archive/macros/latex/contrib/caption/




% *** FLOAT PACKAGES ***
%
%\usepackage{fixltx2e}
% fixltx2e, the successor to the earlier fix2col.sty, was written by
% Frank Mittelbach and David Carlisle. This package corrects a few problems
% in the LaTeX2e kernel, the most notable of which is that in current
% LaTeX2e releases, the ordering of single and double column floats is not
% guaranteed to be preserved. Thus, an unpatched LaTeX2e can allow a
% single column figure to be placed prior to an earlier double column
% figure. The latest version and documentation can be found at:
% http://www.ctan.org/tex-archive/macros/latex/base/



%\usepackage{stfloats}
% stfloats.sty was written by Sigitas Tolusis. This package gives LaTeX2e
% the ability to do double column floats at the bottom of the page as well
% as the top. (e.g., "\begin{figure*}[!b]" is not normally possible in
% LaTeX2e). It also provides a command:
%\fnbelowfloat
% to enable the placement of footnotes below bottom floats (the standard
% LaTeX2e kernel puts them above bottom floats). This is an invasive package
% which rewrites many portions of the LaTeX2e float routines. It may not work
% with other packages that modify the LaTeX2e float routines. The latest
% version and documentation can be obtained at:
% http://www.ctan.org/tex-archive/macros/latex/contrib/sttools/
% Documentation is contained in the stfloats.sty comments as well as in the
% presfull.pdf file. Do not use the stfloats baselinefloat ability as IEEE
% does not allow \baselineskip to stretch. Authors submitting work to the
% IEEE should note that IEEE rarely uses double column equations and
% that authors should try to avoid such use. Do not be tempted to use the
% cuted.sty or midfloat.sty packages (also by Sigitas Tolusis) as IEEE does
% not format its papers in such ways.





% *** PDF, URL AND HYPERLINK PACKAGES ***
%
%\usepackage{url}
% url.sty was written by Donald Arseneau. It provides better support for
% handling and breaking URLs. url.sty is already installed on most LaTeX
% systems. The latest version can be obtained at:
% http://www.ctan.org/tex-archive/macros/latex/contrib/misc/
% Read the url.sty source comments for usage information. Basically,
% \url{my_url_here}.





% *** Do not adjust lengths that control margins, column widths, etc. ***
% *** Do not use packages that alter fonts (such as pslatex).         ***
% There should be no need to do such things with IEEEtran.cls V1.6 and later.
% (Unless specifically asked to do so by the journal or conference you plan
% to submit to, of course. )


% correct bad hyphenation here
\hyphenation{op-tical net-works semi-conduc-tor}

% Macro for certain acronyms in small caps. Doesn't work with the
% default font, though (it contains no smallcaps it seems).
\def\VHDL{\textsc{VHDL}}
\def\GHC{\textsc{GHC}}
\def\CLaSH{C$\lambda$aSH}

% Macro for pretty printing haskell snippets. Just monospaced for now, perhaps
% we'll get something more complex later on.
\def\hs#1{\texttt{#1}}

\begin{document}
%
% paper title
% can use linebreaks \\ within to get better formatting as desired
\title{\CLaSH: Structural Descriptions \\ of Synchronous Hardware using Haskell}


% author names and affiliations
% use a multiple column layout for up to three different
% affiliations
\author{\IEEEauthorblockN{Christiaan P.R. Baaij, Matthijs Kooijman, Jan Kuper, Marco E.T. Gerards, Bert Molenkamp, Sabih H. Gerez}
\IEEEauthorblockA{University of Twente, Department of EEMCS\\
P.O. Box 217, 7500 AE, Enschede, The Netherlands\\
c.p.r.baaij@utwente.nl, matthijs@stdin.nl}}
% \and
% \IEEEauthorblockN{Homer Simpson}
% \IEEEauthorblockA{Twentieth Century Fox\\
% Springfield, USA\\
% Email: homer@thesimpsons.com}
% \and
% \IEEEauthorblockN{James Kirk\\ and Montgomery Scott}
% \IEEEauthorblockA{Starfleet Academy\\
% San Francisco, California 96678-2391\\
% Telephone: (800) 555--1212\\
% Fax: (888) 555--1212}}

% conference papers do not typically use \thanks and this command
% is locked out in conference mode. If really needed, such as for
% the acknowledgment of grants, issue a \IEEEoverridecommandlockouts
% after \documentclass

% for over three affiliations, or if they all won't fit within the width
% of the page, use this alternative format:
% 
%\author{\IEEEauthorblockN{Michael Shell\IEEEauthorrefmark{1},
%Homer Simpson\IEEEauthorrefmark{2},
%James Kirk\IEEEauthorrefmark{3}, 
%Montgomery Scott\IEEEauthorrefmark{3} and
%Eldon Tyrell\IEEEauthorrefmark{4}}
%\IEEEauthorblockA{\IEEEauthorrefmark{1}School of Electrical and Computer Engineering\\
%Georgia Institute of Technology,
%Atlanta, Georgia 30332--0250\\ Email: see http://www.michaelshell.org/contact.html}
%\IEEEauthorblockA{\IEEEauthorrefmark{2}Twentieth Century Fox, Springfield, USA\\
%Email: homer@thesimpsons.com}
%\IEEEauthorblockA{\IEEEauthorrefmark{3}Starfleet Academy, San Francisco, California 96678-2391\\
%Telephone: (800) 555--1212, Fax: (888) 555--1212}
%\IEEEauthorblockA{\IEEEauthorrefmark{4}Tyrell Inc., 123 Replicant Street, Los Angeles, California 90210--4321}}




% use for special paper notices
%\IEEEspecialpapernotice{(Invited Paper)}




% make the title area
\maketitle


\begin{abstract}
%\boldmath
The abstract goes here.
\end{abstract}
% IEEEtran.cls defaults to using nonbold math in the Abstract.
% This preserves the distinction between vectors and scalars. However,
% if the conference you are submitting to favors bold math in the abstract,
% then you can use LaTeX's standard command \boldmath at the very start
% of the abstract to achieve this. Many IEEE journals/conferences frown on
% math in the abstract anyway.

% no keywords




% For peer review papers, you can put extra information on the cover
% page as needed:
% \ifCLASSOPTIONpeerreview
% \begin{center} \bfseries EDICS Category: 3-BBND \end{center}
% \fi
%
% For peerreview papers, this IEEEtran command inserts a page break and
% creates the second title. It will be ignored for other modes.
\IEEEpeerreviewmaketitle


\section{Introduction}
Hardware description languages has allowed the productivity of hardware engineers to keep pace with the development of chip technology. Standard Hardware description languages, like VHDL and Verilog, allowed an engineer to describe circuits using a programming language. These standard languages are very good at describing detailed hardware properties such as timing behavior, but are generally cumbersome in expressing higher-level abstractions. These languages also tend to have a complex syntax and a lack of formal semantics. To overcome these complexities, and raise the abstraction level, a great number of approaches based on functional languages has been proposed \cite{T-Ruby,Hydra,HML2,Hawk1,Lava,ForSyDe1,Wired,reFLect}. The idea of using functional languages started in the early 1980s \cite{Cardelli1981,muFP,DAISY,FHDL}, a time which also saw the birth of the currently popular hardware description languages such as VHDL.

\section{Hardware description in Haskell}

  To translate Haskell to hardware, every Haskell construct needs a
  translation to \VHDL. There are often multiple valid translations
  possible. When faced with choices, the most obvious choice has been
  chosen wherever possible. In a lot of cases, when a programmer looks
  at a functional hardware description it is completely clear what
  hardware is described. We want our translator to generate exactly that
  hardware whenever possible, to make working with \CLaSH\ as intuitive as
  possible.

  \subsection{Function application}
    The basic syntactic elements of a functional program are functions
    and function application. These have a single obvious \VHDL\
    translation: each top level function becomes a hardware component,
    where each argument is an input port and the result value is the
    (single) output port. This output port can have a complex type (such
    as a tuple), so having just a single output port does not pose a
    limitation.

    Each function application in turn becomes component instantiation.
    Here, the result of each argument expression is assigned to a
    signal, which is mapped to the corresponding input port. The output
    port of the function is also mapped to a signal, which is used as
    the result of the application.

    Since every top level function generates its own component, the
    hierarchy of of function calls is reflected in the final \VHDL\
    output as well, creating a hierarchical \VHDL\ description of the
    hardware.  This separation in different components makes the
    resulting \VHDL\ output easier to read and debug.

  \subsection{Choice}
    Although describing components and connections allows us to describe
    a lot of hardware designs already, there is an obvious thing
    missing: choice. We need some way to be able to choose between
    values based on another value.  In Haskell, choice is achieved by
    \hs{case} expressions, \hs{if} expressions, pattern matching and
    guards.

    However, to be able to describe our hardware in a more convenient
    way, we also want to translate Haskell's choice mechanisms. The
    easiest of these are of course case expressions (and \hs{if}
    expressions, which can be very directly translated to \hs{case}
    expressions). A \hs{case} expression can in turn simply be
    translated to a conditional assignment, where the conditions use
    equality comparisons against the constructors in the \hs{case}
    expressions.

    A slightly more complex (but very powerful) form of choice is
    pattern matching. A function can be defined in multiple clauses,
    where each clause specifies a pattern. When the arguments match the
    pattern, the corresponding clause will be used.

  \subsection{Types}
    Translation of two most basic functional concepts has been
    discussed: function application and choice. Before looking further
    into less obvious concepts like higher-order expressions and
    polymorphism, the possible types that can be used in hardware
    descriptions will be discussed.

    Some way is needed to translate every values used to its hardware
    equivalents. In particular, this means a hardware equivalent for
    every \emph{type} used in a hardware description is needed

    Since most functional languages have a lot of standard types that
    are hard to translate (integers without a fixed size, lists without
    a static length, etc.), a number of \quote{built-in} types will be
    defined first. These types are built-in in the sense that our
    compiler will have a fixed VHDL type for these. User defined types,
    on the other hand, will have their hardware type derived directly
    from their Haskell declaration automatically, according to the rules
    sketched here.

  \subsection{Built-in types}
    The language currently supports the following built-in types. Of these,
    only the \hs{Bool} type is supported by Haskell out of the box (the
    others are defined by the \CLaSH\ package, so they are user-defined types
    from Haskell's point of view).

    \begin{description}
      \item[\hs{Bit}]
        This is the most basic type available. It is mapped directly onto
        the \texttt{std\_logic} \VHDL\ type. Mapping this to the
        \texttt{bit} type might make more sense (since the Haskell version
        only has two values), but using \texttt{std\_logic} is more standard
        (and allowed for some experimentation with don't care values)

      \item[\hs{Bool}]
        This is the only built-in Haskell type supported and is translated
        exactly like the Bit type (where a value of \hs{True} corresponds to a
        value of \hs{High}). Supporting the Bool type is particularly
        useful to support \hs{if ... then ... else ...} expressions, which
        always have a \hs{Bool} value for the condition.

        A \hs{Bool} is translated to a \texttt{std\_logic}, just like \hs{Bit}.
      \item[\hs{SizedWord}, \hs{SizedInt}]
        These are types to represent integers. A \hs{SizedWord} is unsigned,
        while a \hs{SizedInt} is signed. These types are parametrized by a
        length type, so you can define an unsigned word of 32 bits wide as
        ollows:

        \begin{verbatim}
          type Word32 = SizedWord D32
        \end{verbatim}

        Here, a type synonym \hs{Word32} is defined that is equal to the
        \hs{SizedWord} type constructor applied to the type \hs{D32}. \hs{D32}
        is the \emph{type level representation} of the decimal number 32,
        making the \hs{Word32} type a 32-bit unsigned word.

        These types are translated to the \small{VHDL} \texttt{unsigned} and
        \texttt{signed} respectively.
      \item[\hs{Vector}]
        This is a vector type, that can contain elements of any other type and
        has a fixed length. It has two type parameters: its
        length and the type of the elements contained in it. By putting the
        length parameter in the type, the length of a vector can be determined
        at compile time, instead of only at run-time for conventional lists.

        The \hs{Vector} type constructor takes two type arguments: the length
        of the vector and the type of the elements contained in it. The state
        type of an 8 element register bank would then for example be:

        \begin{verbatim}
        type RegisterState = Vector D8 Word32
        \end{verbatim}

        Here, a type synonym \hs{RegisterState} is defined that is equal to
        the \hs{Vector} type constructor applied to the types \hs{D8} (The type
        level representation of the decimal number 8) and \hs{Word32} (The 32
        bit word type as defined above). In other words, the
        \hs{RegisterState} type is a vector of 8 32-bit words.

        A fixed size vector is translated to a \VHDL\ array type.
      \item[\hs{RangedWord}]
        This is another type to describe integers, but unlike the previous
        two it has no specific bit-width, but an upper bound. This means that
        its range is not limited to powers of two, but can be any number.
        A \hs{RangedWord} only has an upper bound, its lower bound is
        implicitly zero. There is a lot of added implementation complexity
        when adding a lower bound and having just an upper bound was enough
        for the primary purpose of this type: type-safely indexing vectors.

        To define an index for the 8 element vector above, we would do:

        \begin{verbatim}
        type RegisterIndex = RangedWord D7
        \end{verbatim}

        Here, a type synonym \hs{RegisterIndex} is defined that is equal to
        the \hs{RangedWord} type constructor applied to the type \hs{D7}. In
        other words, this defines an unsigned word with values from
        0 to 7 (inclusive). This word can be be used to index the
        8 element vector \hs{RegisterState} above.

        This type is translated to the \texttt{unsigned} \VHDL type.
    \end{description}
  \subsection{User-defined types}
    There are three ways to define new types in Haskell: algebraic
    data-types with the \hs{data} keyword, type synonyms with the \hs{type}
    keyword and type renamings with the \hs{newtype} keyword. \GHC\
    offers a few more advanced ways to introduce types (type families,
    existential typing, \small{GADT}s, etc.) which are not standard
    Haskell.  These will be left outside the scope of this research.

    Only an algebraic datatype declaration actually introduces a
    completely new type, for which we provide the \VHDL\ translation
    below. Type synonyms and renamings only define new names for
    existing types (where synonyms are completely interchangeable and
    renamings need explicit conversion). Therefore, these do not need
    any particular \VHDL\ translation, a synonym or renamed type will
    just use the same representation as the original type. The
    distinction between a renaming and a synonym does no longer matter
    in hardware and can be disregarded in the generated \VHDL.

    For algebraic types, we can make the following distinction: 

    \begin{description}

      \item[Product types]
        A product type is an algebraic datatype with a single constructor with
        two or more fields, denoted in practice like (a,b), (a,b,c), etc. This
        is essentially a way to pack a few values together in a record-like
        structure. In fact, the built-in tuple types are just algebraic product
        types (and are thus supported in exactly the same way).

        The ``product'' in its name refers to the collection of values belonging
        to this type. The collection for a product type is the Cartesian
        product of the collections for the types of its fields.

        These types are translated to \VHDL\ record types, with one field for
        every field in the constructor. This translation applies to all single
        constructor algebraic data-types, including those with just one
        field (which are technically not a product, but generate a VHDL
        record for implementation simplicity).
      \item[Enumerated types]
        An enumerated type is an algebraic datatype with multiple constructors, but
        none of them have fields. This is essentially a way to get an
        enumeration-like type containing alternatives.

        Note that Haskell's \hs{Bool} type is also defined as an
        enumeration type, but we have a fixed translation for that.

        These types are translated to \VHDL\ enumerations, with one value for
        each constructor. This allows references to these constructors to be
        translated to the corresponding enumeration value.
      \item[Sum types]
        A sum type is an algebraic datatype with multiple constructors, where
        the constructors have one or more fields. Technically, a type with
        more than one field per constructor is a sum of products type, but
        for our purposes this distinction does not really make a
        difference, so this distinction is note made.

        The ``sum'' in its name refers again to the collection of values
        belonging to this type. The collection for a sum type is the
        union of the the collections for each of the constructors.

        Sum types are currently not supported by the prototype, since there is
        no obvious \VHDL\ alternative. They can easily be emulated, however, as
        we will see from an example:

        \begin{verbatim}
        data Sum = A Bit Word | B Word
        \end{verbatim}

        An obvious way to translate this would be to create an enumeration to
        distinguish the constructors and then create a big record that
        contains all the fields of all the constructors. This is the same
        translation that would result from the following enumeration and
        product type (using a tuple for clarity):

        \begin{verbatim}
        data SumC = A | B
        type Sum = (SumC, Bit, Word, Word)
        \end{verbatim}

        Here, the \hs{SumC} type effectively signals which of the latter three
        fields of the \hs{Sum} type are valid (the first two if \hs{A}, the
        last one if \hs{B}), all the other ones have no useful value.

        An obvious problem with this naive approach is the space usage: the
        example above generates a fairly big \VHDL\ type. Since we can be
        sure that the two \hs{Word}s in the \hs{Sum} type will never be valid
        at the same time, this is a waste of space.

        Obviously, duplication detection could be used to reuse a
        particular field for another constructor, but this would only
        partially solve the problem. If two fields would be, for
        example, an array of 8 bits and an 8 bit unsigned word, these are
        different types and could not be shared. However, in the final
        hardware, both of these types would simply be 8 bit connections,
        so we have a 100\% size increase by not sharing these.
      \end{description}


\section{\CLaSH\ prototype}

foo\par bar

\section{Related work}
Many functional hardware description languages have been developed over the years. Early work includes such languages as $\mu$FP~\cite{muFP}, an extension of Backus' FP language to synchronous streams, designed particularly for describing and reasoning about regular circuits. The Ruby~\cite{Ruby} language uses relations, instead of functions, to describe circuits, and has a particular focus on layout. HML~\cite{HML2} is a hardware modeling language based on the strict functional language ML, and has support for polymorphic types and higher-order functions. Published work suggests that there is no direct simulation support for HML, and that the translation to VHDL is only partial.

Like this work, many functional hardware description languages have some sort of foundation in the functional programming language Haskell. Hawk~\cite{Hawk1} uses Haskell to describe system-level executable specifications used to model the behavior of superscalar microprocessors. Hawk specifications can be simulated, but there seems to be no support for automated circuit synthesis. The ForSyDe~\cite{ForSyDe2} system uses Haskell to specify abstract system models, which can (manually) be transformed into an implementation model using semantic preserving transformations. ForSyDe has several simulation and synthesis backends, though synthesis is restricted to the synchronous subset of the ForSyDe language.

Lava~\cite{Lava} is a hardware description language that focuses on the structural representation of hardware. Besides support for simulation and circuit synthesis, Lava descriptions can be interfaced with formal method tools for formal verification. Lava descriptions are actually circuit generators when viewed from a synthesis viewpoint, in that the language elements of Haskell, such as choice, can be used to guide the circuit generation. If a developer wants to insert a choice element inside an actual circuit he will have to specify this explicitly as a component. In this respect \CLaSH\ differs from Lava, in that all the choice elements, such as case-statements and patter matching, are synthesized to choice elements in the eventual circuit. As such, richer control structures can both be specified and synthesized in \CLaSH\ compared to any of the languages mentioned in this section.

The merits of polymorphic typing, combined with higher-order functions, are now also recognized in the `main-stream' hardware description languages, exemplified by the new VHDL 2008 standard~\cite{VHDL2008}. VHDL-2008 has support to specify types as generics, thus allowing a developer to describe polymorphic components. Note that those types still require an explicit generic map, whereas type-inference and type-specialization are implicit in \CLaSH.

Wired~\cite{Wired},, T-Ruby~\cite{T-Ruby}, Hydra~\cite{Hydra}. 

A functional language designed specifically for hardware design is $re{\mathit{FL}}^{ect}$~\cite{reFLect}, which draws experience from earlier language called FL~\cite{FL} to la

% An example of a floating figure using the graphicx package.
% Note that \label must occur AFTER (or within) \caption.
% For figures, \caption should occur after the \includegraphics.
% Note that IEEEtran v1.7 and later has special internal code that
% is designed to preserve the operation of \label within \caption
% even when the captionsoff option is in effect. However, because
% of issues like this, it may be the safest practice to put all your
% \label just after \caption rather than within \caption{}.
%
% Reminder: the "draftcls" or "draftclsnofoot", not "draft", class
% option should be used if it is desired that the figures are to be
% displayed while in draft mode.
%
%\begin{figure}[!t]
%\centering
%\includegraphics[width=2.5in]{myfigure}
% where an .eps filename suffix will be assumed under latex, 
% and a .pdf suffix will be assumed for pdflatex; or what has been declared
% via \DeclareGraphicsExtensions.
%\caption{Simulation Results}
%\label{fig_sim}
%\end{figure}

% Note that IEEE typically puts floats only at the top, even when this
% results in a large percentage of a column being occupied by floats.


% An example of a double column floating figure using two subfigures.
% (The subfig.sty package must be loaded for this to work.)
% The subfigure \label commands are set within each subfloat command, the
% \label for the overall figure must come after \caption.
% \hfil must be used as a separator to get equal spacing.
% The subfigure.sty package works much the same way, except \subfigure is
% used instead of \subfloat.
%
%\begin{figure*}[!t]
%\centerline{\subfloat[Case I]\includegraphics[width=2.5in]{subfigcase1}%
%\label{fig_first_case}}
%\hfil
%\subfloat[Case II]{\includegraphics[width=2.5in]{subfigcase2}%
%\label{fig_second_case}}}
%\caption{Simulation results}
%\label{fig_sim}
%\end{figure*}
%
% Note that often IEEE papers with subfigures do not employ subfigure
% captions (using the optional argument to \subfloat), but instead will
% reference/describe all of them (a), (b), etc., within the main caption.


% An example of a floating table. Note that, for IEEE style tables, the 
% \caption command should come BEFORE the table. Table text will default to
% \footnotesize as IEEE normally uses this smaller font for tables.
% The \label must come after \caption as always.
%
%\begin{table}[!t]
%% increase table row spacing, adjust to taste
%\renewcommand{\arraystretch}{1.3}
% if using array.sty, it might be a good idea to tweak the value of
% \extrarowheight as needed to properly center the text within the cells
%\caption{An Example of a Table}
%\label{table_example}
%\centering
%% Some packages, such as MDW tools, offer better commands for making tables
%% than the plain LaTeX2e tabular which is used here.
%\begin{tabular}{|c||c|}
%\hline
%One & Two\\
%\hline
%Three & Four\\
%\hline
%\end{tabular}
%\end{table}


% Note that IEEE does not put floats in the very first column - or typically
% anywhere on the first page for that matter. Also, in-text middle ("here")
% positioning is not used. Most IEEE journals/conferences use top floats
% exclusively. Note that, LaTeX2e, unlike IEEE journals/conferences, places
% footnotes above bottom floats. This can be corrected via the \fnbelowfloat
% command of the stfloats package.



\section{Conclusion}
The conclusion goes here.




% conference papers do not normally have an appendix


% use section* for acknowledgement
\section*{Acknowledgment}


The authors would like to thank...





% trigger a \newpage just before the given reference
% number - used to balance the columns on the last page
% adjust value as needed - may need to be readjusted if
% the document is modified later
%\IEEEtriggeratref{8}
% The "triggered" command can be changed if desired:
%\IEEEtriggercmd{\enlargethispage{-5in}}

% references section

% can use a bibliography generated by BibTeX as a .bbl file
% BibTeX documentation can be easily obtained at:
% http://www.ctan.org/tex-archive/biblio/bibtex/contrib/doc/
% The IEEEtran BibTeX style support page is at:
% http://www.michaelshell.org/tex/ieeetran/bibtex/
\bibliographystyle{IEEEtran}
% argument is your BibTeX string definitions and bibliography database(s)
\bibliography{IEEEabrv,cλash.bib}
%
% <OR> manually copy in the resultant .bbl file
% set second argument of \begin to the number of references
% (used to reserve space for the reference number labels box)
% \begin{thebibliography}{1}
% 
% \bibitem{IEEEhowto:kopka}
% H.~Kopka and P.~W. Daly, \emph{A Guide to \LaTeX}, 3rd~ed.\hskip 1em plus
%   0.5em minus 0.4em\relax Harlow, England: Addison-Wesley, 1999.
% 
% \end{thebibliography}




% that's all folks
\end{document}

% vim: set ai sw=2 sts=2 expandtab:
